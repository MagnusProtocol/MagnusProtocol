\documentclass{article}

\title{The Magnus Protocol}
\author{SteveGremory}

\usepackage{helvet}

\begin{document}

\maketitle

\begin{abstract}
This paper describes an algorithm for transferring files between two computers.
We assume that the two computers are connected to each other by the internet (IPv4)
with no apparent packet loss. The algorithm, in it's current state, is basic in nature
and easy to understand. The algorithm is explained later in the paper.
\end{abstract}

\section{The problem}

Today, if two people need to exchange a file with each other, they
need to go through useless hassles like uploading it to a random website
which they do not trust/should not trust or uploading it to some paid cloud
provider's platform. \\


People don't really have much of a choice, it's either trusting
someone seemingly random individual or company with your data and in some
cases, paying for the service to get larger upload sizes. If not this, then they
are probably going to use some CLI application that may be hard to operate for
regular GUI users. \\


This is the problem Magnus aims to solve. Magnus' only forseeable goal is to make
file transfer between two computers over the internet fast, efficeint and easy. \\


Magnus also aims to make use of UDP to make the file trasfers faster than
traditional TCP file transfers and have a well designed, simple and fast GUI
for Linux and macOS (with Windows support soon to come). It also aims to be
secure and end-to-end encrypt every single transfer and verify every single
part sent over by comparing their hashes. \\

\section{The way it works}

\section{``It's in the way that you use it''}
If implemented correctly, Magnus has the potential of being a great tool, but
ultimately, it will always be up to the user to decide what to do with it. \\

We value privacy more than anything (even speed), making it none of our (the developers')
or anyone elses' business what you use it for; however, we sill suggest that
you to not use Magnus for anything immoral. \\

The software is licenced under GPLv3 and whatever the user does with it
is their responsibility. The developers take no responsibility over how
the software is being used and are not concerned with it either as long
as the GPL licence isn't being violated.

\end{document}
