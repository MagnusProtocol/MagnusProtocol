\documentclass{article}

\title{The Magnus Protocol}
\author{SteveGremory}

\usepackage{helvet}

\begin{document}

\maketitle

\begin{abstract}
      This paper describes a file exchange protocol for transferring files between two computers.
      We assume that the two computers are connected to each other by the internet (IPv4)
      with no apparent packet loss. The protocol, in it's current state, is basic in nature
      and easy to understand. The protocol is explained later in the paper.
\end{abstract}

\section{The problem}

Today, if two people need to exchange a file with each other, they need to go
through useless hassles like uploading it to a random website which they do not
trust/should not trust or uploading it to some paid cloud provider's platform.
\\

People don't really have much of a choice, it's either trusting someone
seemingly random individual or company with your data and in some cases, paying
for the service to get larger upload sizes. If not this, then they are probably
going to use some CLI application that may be hard to operate for regular GUI
users. \\

This is the problem Magnus aims to solve. Magnus' only forseeable goal is to
make file transfer between two computers over the internet fast, efficient and
easy. \\

Magnus also aims to make use of UDP to make the file transfers faster than
traditional TCP file transfers and have a well designed, simple and fast GUI
for Linux and macOS (with Windows support soon to come). It also aims to be
secure and end-to-end encrypt every single transfer and verify every single
part sent over by comparing their hashes. \\

\section{The way it works}

Please keep in mind that Magnus is still pre-alpha. Everything written here is
subject to change at any point in time. \\

Consider two computers (A and B) connected to each other via the internet
(IPv4) and have the required ports, whatever they might be, opened to the
internet. Let's say that computer A wants to send a file, let's say image.jpg
over to computer B using Magnus. \\

To proceed with this, computer A will need to have read access to ``image.jpg''
and computer B will need to have write access to the filesystem. Both computers
will need to have a Magnus instance running, and know their public IP
addresses. \\

When all of these requirements are met, the following sequence of actions will
take place:
\begin{enumerate}
      \item When a transfer is initiated, a TCP socket connection will be opened between
            computer A and B.
      \item Both computers will generate new private/public key-pairs using ECDH (Elliptic
            Curve Diffie-Hellman) and send each other their public keys over the TCP
            connection opened earlier. After the keys have been exchanged, both computers
            will generate shared keys using the aforementioned.
      \item Computer A will now read the file from the disk and load it into memory by
            streaming it from disk. As each part is read from the disk, it will be put in a
            queue to be sent over to computer B. The size of each packet read from the disk
            and the queue shall be implementation dependant.
      \item As the previous task is being carried out by the main thread, another thread
            will be consuming the queue and
            \textbf{labelling/hashing/encrypting/compressing} the parts and eventually
            writing all the information to a csv file. This will include: the size of the
            packet, it's hash, it's label number and the data, where data can be a pointer
            to the data in memory or some of the data itself. Ideally, all of these
            processes should be delegated to different threads.
      \item The label and the size of the packet will be added at the top of each packet
            alongside other header information.
      \item Now, the csv file mentioned in the last step will be sent over the TCP
            connection and a UDP connection will be established between both the computers
            to facilitate the transfer of the actual data. This is where Magnus stands out,
            the transfer process will be explained in the next section of the paper.
      \item Once computer B has ensured that all the parts have reached the destination, it
            will send back a message to computer A, concluding the transfer.
\end{enumerate}

\section{The UDP Transfer}
The UDP transfer is what makes Mangus different from other widely available
protocols. It isn't being claimed that this hasn't been done before and no
research was conducted by the authors regarding this topic. \\

Here's how it works in a nutshell, implementations may modify this process to
suit their needs:
\begin{enumerate}
      \item First, all the parts are sent over the UDP socket one by one.
      \item As the receiver gets each part, it looks at the part's header and matches it
            with the information contained in the CSV that was sent over TCP before the
            actual data packet transfer began. In the worst case, the client will keep
            waiting for the file to fully arrive until the transfer ends.
      \item As soon as a file matching the size and the header is completely received, it
            will be hashed and it's hash will be compared to the hash sent by the sender.
            If the hash matches, the file packet will be written to disk. If not, it's
            label, size and hash will be stored in memory/written to a file on disk for
            later use which will be elaborated in the points to come.
      \item The client will have to do this in a non-blocking fashion, this will be
            implementation dependant.
      \item Once the first round of packets have been sent, the sender will send an
            acknowledgement to the client suggesting that all the packets have been sent at
            least once.
      \item Now, if any packets didn't match up, the client will request receiver to send
            over those packets again and this process will recursively continue until all
            the packets have been received.
\end{enumerate}

\section{``It's in the way that you use it''}
If implemented correctly, Magnus has the potential of being a great tool, but
ultimately, it will always be up to the user to decide what to do with it. \\

We value privacy more than anything (even speed), making it none of our (the
developers') or anyone elses' business what you use it for; however, we sill
suggest that you to not use Magnus for anything immoral. \\

The software is licensed under GPLv3 and whatever the user does with it is
their responsibility. The developers take no responsibility over how the
software is being used and are not concerned with it either as long as the GPL
license isn't being violated.

\end{document}