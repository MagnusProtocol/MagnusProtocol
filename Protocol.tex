\documentclass{article}

\title{The Magnus Protocol}
\author{SteveGremory}

\usepackage{helvet}

\begin{document}

\maketitle

\begin{abstract}
This paper describes a file exchange protocol for transferring files between two computers.
We assume that the two computers are connected to each other by the internet (IPv4)
with no apparent packet loss. The protocol, in it's current state, is basic in nature
and easy to understand. The protocol is explained later in the paper.
\end{abstract}

\section{The problem}

Today, if two people need to exchange a file with each other, they
need to go through useless hassles like uploading it to a random website
which they do not trust/should not trust or uploading it to some paid cloud
provider's platform. \\


People don't really have much of a choice, it's either trusting
someone seemingly random individual or company with your data and in some
cases, paying for the service to get larger upload sizes. If not this, then they
are probably going to use some CLI application that may be hard to operate for
regular GUI users. \\


This is the problem Magnus aims to solve. Magnus' only forseeable goal is to make
file transfer between two computers over the internet fast, efficeint and easy. \\


Magnus also aims to make use of UDP to make the file trasfers faster than
traditional TCP file transfers and have a well designed, simple and fast GUI
for Linux and macOS (with Windows support soon to come). It also aims to be
secure and end-to-end encrypt every single transfer and verify every single
part sent over by comparing their hashes. \\

\section{The way it works}

Please keep in mind that Magnus is still pre-alpha.
Everything written here is subject to change at any point in time. \\


Consider two computers (A and B) connected to each other via the internet (IPv4)
and have the required ports, whatever they might be, opened to the internet. Let's
say that computer A wants to send a file, let's say image.jpg over to computer B using Magnus. \\


To proceed with this, computer A will need to have read access to ``image.jpg'' and
computer B will need to have write access to the filesystem. Both computers will need
to have a Magnus instance running, and know their public IP addresses. \\


When all of these requirements are met, the following sequence of actions will take place:
\begin{enumerate}
  \item When a transfer is initiated, a TCP socket connection will be opened between
        computer A and B.
  \item Both computers will gererate new private/public keypairs using ECDH
        (Elliptic Curve Diffie-Hellman) and send each other their public keys over
        the TCP connection opened earlier. After the keys have been exchanged, both
        computers will generate shared keys using the Diffie-Hellman key exchange method.
  \item Computer A will now read the file from the disk and load it into
        memory by mapping it in memory. If the file is too big to be mapped at
        once, it will be divided into parts and those parts will go through the same process.
  \item Once in memory, the file will be hashed (ideally, using the BLAKE3 hashing algorithm)
        and encrypted (ideally, using a secure AES256 implementation).
  \item The file, now encrypted, will be compressed (ideally, using ZSTD) to reduce overall data usage.
  \item The compressed data will now be cut into X non-overlapping parts of Y sizes where
        X and Y will be determined by the ideal packet size function which is explained later in this paper.
  \item Now, a UDP connection will be established between both the computers and the file will be sent
        over it. This process is the key part of what makes Magnus special and is also explained later
        in the paper.
  \item Once computer B has ensured that all the parts have reached the destination, it will send back a
        message to computer A, condluding the transfer.

\end{enumerate}


\section{``It's in the way that you use it''}
If implemented correctly, Magnus has the potential of being a great tool, but
ultimately, it will always be up to the user to decide what to do with it. \\

We value privacy more than anything (even speed), making it none of our (the developers')
or anyone elses' business what you use it for; however, we sill suggest that
you to not use Magnus for anything immoral. \\

The software is licenced under GPLv3 and whatever the user does with it
is their responsibility. The developers take no responsibility over how
the software is being used and are not concerned with it either as long
as the GPL licence isn't being violated.

\end{document}
